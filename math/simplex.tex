\subsection{线性规划}
	\begin{lstlisting}[language=c++]
#define MAXM 20  //max num of basic varibles
#define INF 1E200

double A[MAXM+5][MAXN+MAXM+5];
double b[MAXM+5],c[MAXN+MAXM+5];
int N[MAXN+5],B[MAXM+5];
double X[MAXN+MAXM+5],V;
int n,m,R,C,nCnt,bCnt;
int v1[MAXN],v2[MAXN];

int fcmp(double a,double b)
{
	if(fabs(a-b)<1E-7) return 0;
	if(a>b) return 1;
	return -1;
}

void Pivot(int l,int e)
{
	double t=A[l][e],p=c[e];
	b[l]=b[l]/t;
	for(int i=1;i<=C;i++)
		A[l][i]/=t;
	V=V-c[e]*b[l];
	for(int i=1;i<=R;i++)
	{
		if(i==l||fcmp(A[i][e],0.0)==0)
			continue;
		t=A[i][e];
		b[i]=b[i]-t*b[l];
		for(int j=1;j<=C;j++)
			A[i][j]=A[i][j]-t*A[l][j];
	}
	for(int i=1;i<=C;i++)
		c[i]=c[i]-p*A[l][i];
	for(int i=1;i<=nCnt;i++)
	{
		if(N[i]==e)
		{
			N[i]=B[l];
			break;
		}
	}
	B[l]=e;
}

bool Process(double P[])
{
	while(true)
	{
		int e=-1;
		double mV=-INF;
		for(int i=1;i<=nCnt;i++)
			if(fcmp(P[N[i]],mV)==1)
				mV=P[N[i]],e=N[i];

		if(fcmp(mV,0.0)<=0) break;
		int l=-1;
		mV=INF;
		for(int i=1;i<=bCnt;i++)
		{
			if(fcmp(A[i][e],0.0)==1)
			{
				double t=b[i]/A[i][e];
				if(fcmp(mV,t)==1||(fcmp(mV,t)==0&&(l==-1||B[l]>B[i])))
					mV=t,l=i;
			}
		}
		if(l==-1) return false;
		Pivot(l,e);
	}
	return true;
}

bool initSimplex()
{
	nCnt=bCnt=0;
	for(int i=1;i<=n;i++)
		N[++nCnt]=i;
	for(int i=1;i<=m;i++)
		B[++bCnt]=i+n,A[i][n+i]=1.0;
	R=bCnt,C=bCnt+nCnt;
	double minV=INF;
	int p=-1;
	for(int i=1;i<=m;i++)
		if(fcmp(minV,b[i])==1)
			minV=b[i],p=i;
	if(fcmp(minV,0.0)>=0)
		return true;
	N[++nCnt]=n+m+1;R++,C++;
	for(int i=0;i<=C;i++)
		A[R][i]=0.0;
	for(int i=1;i<=R;i++)
		A[i][n+m+1]=-1.0;
	Pivot(p,n+m+1);
	if(!Process(A[R])) return false;
	if(fcmp(b[R],0.0)!=0)
		return false;
	p=-1;
	for(int i=1;i<=bCnt&&p==-1;i++)
		if(B[i]==n+m+1) p=i;
	if(p!=-1)
	{
		for(int i=1;i<=nCnt;i++)
		{
			if(fcmp(A[p][N[i]],0.0)!=0)
			{
				Pivot(p,N[i]);
				break;
			}
		}
	}
	bool f=false;
	for(int i=1;i<=nCnt;i++)
	{
		if(N[i]==n+m+1) f=true;
		if(f&&i+1<=nCnt)
			N[i]=N[i+1];
	}
	nCnt--;
	R--,C--;
	return true;
}

//-1: no solution 1: no bound 0: has a solution -V
int Simplex()
{
	if(!initSimplex())
		return -1;
	if(!Process(c))
		return 1;
	for(int i=1;i<=nCnt;i++)
		X[N[i]]=0.0;
	for(int i=1;i<=bCnt;i++)
		X[B[i]]=b[i];
	return 0;
}

int main()
{
	//n = 1;m=1;
	//V= 0.0;
	//c[1] = 1.0;
	//A[1][1] = 1.0;
	//b[1] = 5.0;
	//Simplex();
	//printf("V = %.3f\n",V);

	while(scanf("%d",&v1[1]) == 1)
		{
			for(int i = 2; i<=6;i++)
				scanf("%d",&v1[i]);
			n = 4; m = 6;
			for(int i = 0 ; i<=m+1;i++)
				for(int j=0;j<=n+m+2;j++)
					A[i][j] = c[j] = 0;
			memset(b,0,sizeof(b));
			V = 0.0;
			/*
			`n 为未知数个数`
			`m 为约束个数`
			`目标:siama(c[i]*xi)`
			`约束:sigma(A[i][j]*xj) <=b[i]; j = 1 ... n`
			`解存在X里面`
			*/
			b[1] = v1[1] ; A[1][1] = 1;A[1][4] = 1;
			b[2] = v1[2] ; A[2][1] = 1;A[2][3] = 1;
			b[3] = v1[3] ; A[3][3] = 1;A[3][4] = 1;
			b[4] = v1[4] ; A[4][2] = 1;A[4][3] = 1;
			b[5] = v1[5] ; A[5][2] = 1;A[5][4] = 1;
			b[6] = v1[6] ; A[6][1] = 1;A[6][2] = 1;
			c[1] = 1;c[2] = 1;c[3] = 1;c[4] = 1;
			Simplex();
			//printf("V = %.3f\n",V);
			printf("%.3f %.3f %.3f %.3f\n",X[1],X[2],X[3],X[4]);

		}
	return 0;
}
	\end{lstlisting}
