\subsection{FFT}
	\subsection{位运算}
$$tf(X1,X2) = (tf(X1) - tf(X2), tf(X1) + tf(X2))$$

异或:$tf(X1,X2) = (tf(X1) - tf(X2), tf(X1) + tf(X2))$ 

与:$tf(x1,x2)=(tf(x1) + tf(x2), tf(x1))$
	\begin{lstlisting}[language=c++]
// Transforms the interval [x, y) in a.
    void transform(int x, int y)
    {
        if ( x == y - 1) {
            return;
        }
        int l2 = ( y - x ) / 2;
        int z = x + l2;
        transform(x, z);
        transform(z, y);
        for (int i=x; i<z; i++) {
            int x1 = a[i];
            int x2 = a[i+l2];
            a[i] = (x1 - x2 + MOD) % MOD;
            a[i+l2] = (x1 + x2) % MOD;
        }
    }
    // Reverses the transform in
    // the interval [x, y) in a.
    void untransform(int x, int y)
    {
        if ( x == y - 1) {
            return;
        }
        int l2 = ( y - x ) / 2;
        int z = x + l2;
        for (int i=x; i<z; i++) {
            long long y1 = a[i];
            long long y2 = a[i+l2];
            // x1 - x2 = y1
            // x1 + x2 = y2
            // 2 * x1  = y1 + y2
            // 2 * x2  = y2 - y1
            
            // In order to solve those equations, we need to divide by 2
            // But we are performing operations modulo 1000000007
            // that needs us to find the modular multiplicative inverse of 2.
            // That is saved in the INV2 variable.
            
            a[i] = (int)( ((y1 + y2)*INV2) % MOD );
            a[i+l2] = (int)( ((y2 - y1 + MOD)*INV2) % MOD );
        }
        untransform(x, z);
        untransform(z, y);
    }
	\end{lstlisting}

	\begin{lstlisting}[language=c++]
const double PI= acos(-1.0);
struct vir
{
	double re,im; //实部和虚部
	vir(double a=0,double b=0)
	{
		re=a;
		im=b;
	}
	vir operator +(const vir &b)
	{return vir(re+b.re,im+b.im);}
	vir operator -(const vir &b)
	{return vir(re-b.re, im-b.im);}
	vir operator *(const vir &b)
	{return vir(re*b.re-im*b.im , re*b.im+im*b.re);}
};
vir x1[200005],x2[200005];
void change(vir *x,int len,int loglen)
{
	int i,j,k,t;
	for(i=0;i<len;i++)
	{
		t=i;
		for(j=k=0; j<loglen; j++,t>>=1)
			k= (k<<1)|(t&1);
		if(k<i)
		{
		//	printf("%d %d\n",k,i);
			vir wt=x[k];
			x[k]=x[i];
			x[i]=wt;
		}
	}
}
void fft(vir *x,int len,int loglen)
{
	int i,j,t,s,e;
	change(x,len,loglen);
	t=1;
	for(i=0;i<loglen;i++,t<<=1)
	{
		s=0;
		e=s+t;
		while(s<len)
		{
			vir a,b,wo(cos(PI/t),sin(PI/t)),wn(1,0);
			for(j=s;j<s+t;j++)
			{
				a=x[j];
				b=x[j+t]*wn;
				x[j]=a+b;
				x[j+t]=a-b;
				wn=wn*wo;
			}
			s=e+t;
			e=s+t;
		}
	}
}
void dit_fft(vir *x,int len,int loglen)
{
	int i,j,s,e,t=1<<loglen;
	for(i=0;i<loglen;i++)
	{
		t>>=1;
		s=0;
		e=s+t;
		while(s<len)
		{
			vir a,b,wn(1,0),wo(cos(PI/t),-sin(PI/t));
			for(j=s;j<s+t;j++)
			{
				a=x[j]+x[j+t];
				b=(x[j]-x[j+t])*wn;
				x[j]=a;
				x[j+t]=b;
				wn=wn*wo;
			}
			s=e+t;
			e=s+t;
		}
	}
	change(x,len,loglen);
	for(i=0;i<len;i++)
		x[i].re/=len;
}
int main()
{
	char a[100005],b[100005];
	int i,len1,len2,len,loglen;
	int t,over;
	while(scanf("%s%s",a,b)!=EOF)
	{
		len1=strlen(a)<<1;
		len2=strlen(b)<<1;
		len=1;loglen=0;
		while(len<len1)
		{
			len<<=1;	loglen++;
		}
		while(len<len2)
		{
			len<<=1;	loglen++;
		}
		for(i=0;a[i];i++)
		{
			x1[i].re=a[i]-'0';
			x1[i].im=0;
		}
		for(;i<len;i++)
			x1[i].re=x1[i].im=0;
		for(i=0;b[i];i++)
		{
			x2[i].re=b[i]-'0';
			x2[i].im=0;
		}
		for(;i<len;i++)
			x2[i].re=x2[i].im=0;
		fft(x1,len,loglen);
		fft(x2,len,loglen);
		for(i=0;i<len;i++)
			x1[i] = x1[i]*x2[i];
		dit_fft(x1,len,loglen);
		for(i=(len1+len2)/2-2,over=len=0;i>=0;i--)
		{
			t=(int)(x1[i].re+over+0.5);
			a[len++]= t%10;
			over = t/10;
		}
		while(over)
		{
			a[len++]=over%10;
			over/=10;
		}
		for(len--;len>=0&&!a[len];len--);
			if(len<0)
			putchar('0');
			else
				for(;len>=0;len--)
					putchar(a[len]+'0');
		putchar('\n');
	}
	return 0;
}
	\end{lstlisting}
