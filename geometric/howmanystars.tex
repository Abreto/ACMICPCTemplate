\subsection{三角形内点个数}
	\subsubsection{无三点共线}
	\begin{lstlisting}[language=c++]
Point p[1000], tp[2000], base;

bool cmp(const Point &a, const Point &b)
{
	return a.theta < b.theta;
}

int cnt[1000][1000];
int cntleft[1000][1000];
int n, m;

int calc(int a, int b, int c)
{
	Point p1 = p[b] - p[a], p2 = p[c] - p[a];
	if (atan2(p1.y, p1.x) > atan2(p2.y, p2.x))
		swap(b, c);
	if ((p[b] - p[a]) * (p[c] - p[a]) > 0)
		return cnt[a][c] - cnt[a][b] - 1;
	else
		return n - 3 - (cnt[a][c] - cnt[a][b] - 1);
}

int main(int argc, char const *argv[])
{
	int totcas;
	scanf("%d", &totcas);
	for (int cas = 1; cas <= totcas; ++cas)
	{
		scanf("%d", &n);
		for (int i = 0; i < n; ++i)
		{
			scanf("%lld%lld", &p[i].x, &p[i].y);
			p[i].id = i;
		}
		for (int i = 0; i < n; ++i)
		{
			m = 0;
			base = p[i];
			for (int j = 0; j < n; ++j)
				if (i != j)
				{
					tp[m] = p[j];
					Point v = tp[m]-base;
					tp[m++].theta = atan2(v.y,v.x);
				}

			sort(tp, tp + m, cmp);
			for (int j = 0; j < m; ++j)
				tp[m + j] = tp[j];

			//calc cnt
			for (int j = 0; j < m; ++j)
				cnt[i][tp[j].id] = j;

			//calc cntleft
			for (int j = 0, k = 0, tot = 0; j < m; ++j)
			{
				while (k == j || (k < j + m && (tp[j] - base) * (tp[k] - base) > 0))
					k++, tot++;
				cntleft[i][tp[j].id] = --tot;
			}
		}
		
		printf("Case %d:\n", cas);
		int q;
		scanf("%d", &q);
		for (int i = 0; i < q; ++i)
		{
			int x, y, z;
			scanf("%d%d%d", &x, &y, &z);
			if ((p[z] - p[x]) * (p[y] - p[x]) > 0)
				swap(y, z);
			int res = cntleft[x][z] + cntleft[z][y] + cntleft[y][x];
			res += calc(x, y, z) + calc(y, z, x) + calc(z, x, y);
			res -= 2 * (n - 3);
			printf("%d\n", res);
		}
	}
	return 0;
}
	\end{lstlisting}
	
	\subsubsection{有三点共线且点有类别之分}
	\begin{lstlisting}[language=c++]
int n,n0,n1,m;
Point p[3000], tp[3000], base;

bool cmp(const Point &a, const Point &b)
{
	if ((a-base)*(b-base) == 0)
	{
		return (a-base).getMol() < (b-base).getMol();
	}
	return a.theta < b.theta;
}

int cnt[100][100];
int cntleft[100][100];

int calc(int a,int b,int c)
{
	Point p1 = p[b]-p[a],p2 = p[c]-p[a];
	if (atan2(1.0*p1.y,1.0*p1.x) > atan2(1.0*p2.y,1.0*p2.x))
		swap(b,c);
	int res = cnt[a][c]-cnt[a][b];
	if ((p[b]-p[a])*(p[c]-p[a]) > 0)
		return res;
	else
		return n1-res;
}

int main()
{
	int cas = 0;
	while (scanf("%d%d",&n0,&n1) != EOF)
	{
		n = n1+n0;
		for (int i = 0; i < n; i++)
		{
			scanf("%I64d%I64d",&p[i].x,&p[i].y);
			p[i].id = i;
		}
		for (int i = 0; i < n0; ++i)
		{
			m = 0;
			base = p[i];
			for (int j = 0; j < n; ++j)
				if (i != j)
				{
					tp[m] = p[j];
					Point v = tp[m]-base;
					tp[m++].theta = atan2(1.0*v.y,1.0*v.x);
				}

			sort(tp, tp + m, cmp);
			for (int j = 0; j < m; ++j)
				tp[m + j] = tp[j];

			for (int j = 0,tot = 0; j < m; ++j)
			{
				if (tp[j].id < n0)
					cnt[i][tp[j].id] = tot;
				else
					tot++;
			}

			for (int j = 0, k = 0, tot = 0; j < m; ++j)
			{
				while (k == j || (k < j + m && (tp[j] - base) * (tp[k] - base) > 0))
				{
					if (tp[k].id >= n0)
						tot++;
					k++;
				}
				if (tp[j].id >= n0)
					tot--;
				else
					cntleft[i][tp[j].id] = tot;
			}
		}

		int ans = 0;
		for (int i = 0; i < n0; i++)
			for (int j = i+1; j < n0; j++)
				for (int k = j+1; k < n0; k++)
				{
					int x = i,y = j,z = k;

					if ((p[z] - p[x]) * (p[y] - p[x]) > 0)
						swap(y, z);
					int res = cntleft[x][z] + cntleft[z][y] + cntleft[y][x];

					res += calc(x, y, z) + calc(y, z, x) + calc(z, x, y);

					res -= 2 * n1;

					//printf("%d %d %d %d\n",x,y,z,res);

					if (res%2 == 1)
						ans++;
				}
		printf("Case %d: %d\n",++cas,ans);
	}
	return 0;
}
	\end{lstlisting}
