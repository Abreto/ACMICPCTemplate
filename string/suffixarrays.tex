\subsection{后缀数组}
	\subsubsection{DC3}
		所有下标都是0~n-1,height[0]无意义。
		\begin{lstlisting}[language=c++]
//所有相关数组都要开三倍
const int maxn = 300010;
# define F(x) ((x)/3+((x)%3==1?0:tb))
# define G(x) ((x)<tb?(x)*3+1:((x)-tb)*3+2)
int wa[maxn * 3], wb[maxn * 3], wv[maxn * 3], ws[maxn * 3];
int c0(int *r, int a, int b)
{
	return 
	r[a] == r[b] && r[a + 1] == r[b + 1] && r[a + 2] == r[b + 2];
}
int c12(int k, int *r, int a, int b)
{
	if (k == 2) 
		return r[a] < r[b] || r[a] == r[b] && c12(1, r, a + 1, b + 1);
	else return r[a] < r[b] || r[a] == r[b] && wv[a + 1] < wv[b + 1];
}
void sort(int *r, int *a, int *b, int n, int m)
{
	int i;
	for (i = 0; i < n; i++) wv[i] = r[a[i]];
	for (i = 0; i < m; i++) ws[i] = 0;
	for (i = 0; i < n; i++) ws[wv[i]]++;
	for (i = 1; i < m; i++) ws[i] += ws[i - 1];
	for (i = n - 1; i >= 0; i--) b[--ws[wv[i]]] = a[i];
	return;
}
void dc3(int *r, int *sa, int n, int m)
{
	int i, j, *rn = r + n;
	int *san = sa + n, ta = 0, tb = (n + 1) / 3, tbc = 0, p;
	r[n] = r[n + 1] = 0;
	for (i = 0; i < n; i++) if (i % 3 != 0) wa[tbc++] = i;
	sort(r + 2, wa, wb, tbc, m);
	sort(r + 1, wb, wa, tbc, m);
	sort(r, wa, wb, tbc, m);
	for (p = 1, rn[F(wb[0])] = 0, i = 1; i < tbc; i++)
		rn[F(wb[i])] = c0(r, wb[i - 1], wb[i]) ? p - 1 : p++;
	if (p < tbc) dc3(rn, san, tbc, p);
	else for (i = 0; i < tbc; i++) san[rn[i]] = i;
	for (i = 0; i < tbc; i++) if (san[i] < tb) wb[ta++] = san[i] * 3;
	if (n % 3 == 1) wb[ta++] = n - 1;
	sort(r, wb, wa, ta, m);
	for (i = 0; i < tbc; i++) wv[wb[i] = G(san[i])] = i;
	for (i = 0, j = 0, p = 0; i < ta && j < tbc; p++)
		sa[p] = c12(wb[j] % 3, r, wa[i], wb[j]) ? wa[i++] : wb[j++];
	for (; i < ta; p++) sa[p] = wa[i++];
	for (; j < tbc; p++) sa[p] = wb[j++];
}
//`str和sa也要三倍`
void da(int str[],int sa[],int rank[],int height[],int n,int m)
{
	for (int i = n; i < n * 3; i++)
		str[i] = 0;
	dc3 (str , sa , n + 1 , m);
	int i, j, k;
	for (i = 0; i < n; i++)
	{
		sa[i] = sa[i + 1];
		rank[sa[i]] = i;
	}
	for (i = 0, j = 0, k = 0; i < n; height[rank[i ++]] = k)
		if (rank[i] > 0)
			for (k ? k-- : 0 , j = sa[rank[i] - 1]; 
				i + k < n && j + k < n && str[i + k] == str[j + k];
				k++);
}
		\end{lstlisting}
		
	\subsubsection{DA}
		这份似乎就没啥要注意的了。
		\begin{lstlisting}[language=c++]
const int maxn = 200010;
int wx[maxn],wy[maxn],*x,*y,wss[maxn],wv[maxn];

bool cmp(int *r,int n,int a,int b,int l)
{
	return a+l<n && b+l<n && r[a]==r[b]&&r[a+l]==r[b+l];
}
void da(int str[],int sa[],int rank[],int height[],int n,int m)
{
	int *s = str;
	int *x=wx,*y=wy,*t,p;
	int i,j;
	for(i=0; i<m; i++)wss[i]=0;
	for(i=0; i<n; i++)wss[x[i]=s[i]]++;
	for(i=1; i<m; i++)wss[i]+=wss[i-1];
	for(i=n-1; i>=0; i--)sa[--wss[x[i]]]=i;
	for(j=1,p=1; p<n && j<n; j*=2,m=p)
	{
		for(i=n-j,p=0; i<n; i++)y[p++]=i;
		for(i=0; i<n; i++)if(sa[i]-j>=0)y[p++]=sa[i]-j;
		for(i=0; i<n; i++)wv[i]=x[y[i]];
		for(i=0; i<m; i++)wss[i]=0;
		for(i=0; i<n; i++)wss[wv[i]]++;
		for(i=1; i<m; i++)wss[i]+=wss[i-1];
		for(i=n-1; i>=0; i--)sa[--wss[wv[i]]]=y[i];
		for(t=x,x=y,y=t,p=1,i=1,x[sa[0]]=0; i<n; i++)
			x[sa[i]]=cmp(y,n,sa[i-1],sa[i],j)?p-1:p++;
	}
	for(int i=0; i<n; i++) rank[sa[i]]=i;
	for(int i=0,j=0,k=0; i<n; height[rank[i++]]=k)
		if(rank[i]>0)
			for(k?k--:0,j=sa[rank[i]-1];
				i+k < n && j+k < n && str[i+k]==str[j+k];
				k++);
}
		\end{lstlisting}
		
	\subsubsection{调用}
		注意几个数组的下标是不同的
		\begin{lstlisting}[language=c++]
char s[maxn];
int str[maxn],sa[maxn],rank[maxn],height[maxn];

int main()
{
	scanf("%s",s);
	int len = strlen(s);
	for (int i = 0;i <= len;i++)
		str[i] = s[i];
	da(str,sa,rank,height,len,128);
	
	for (int i = 0;i < len;i++)
	{
		printf("sa= %d,height= %d,s= %s\n",sa[i],height[i],s+sa[i]);
	}
	return 0;
}
		\end{lstlisting}
		
	\subsubsection{最长公共前缀}
		记得不要忘记调用lcpinit!
		\begin{lstlisting}[language=c++]
int f[maxn][20];
int lent[maxn];
void lcpinit()
{
	int i,j;
	int n = len,k = 1,l = 0;
	for (i = 0; i < n; i++)
	{
		f[i][0] = height[i];
		if (i+1 > k*2)
		{
			k *= 2;
			l++;
		}
		lent[i+1] = l;
	}
	for (j = 1; (1<<j)-1<n; j++)
		for (i = 0; i+(1<<j)-1<n; i++)
			f[i][j] = min(f[i][j-1],f[i+(1<<(j-1))][j-1]);
}
int lcp(int x,int y)
{
	if (x > y)	swap(x,y);
	if (x == y)
		return x-sa[x];//自己和自己lcp当然是自己的长度啦
	x++;
	int k = lent[y-x+1];
	return min(f[x][k],f[y-(1<<k)+1][k]);
}
		\end{lstlisting}
		
	\subsubsection{最长公共前缀大于等于某个值的区间}
		\begin{lstlisting}[language=c++]
void getinterv(int pos,int comlen,int& pl,int& pr)
{
	int l,r,mid,cp;
	l = 0;
	r = pos;
	while (l < r)
	{
		mid = l+r>>1;
		cp = lcp(mid,pos);
		if (cp < comlen)
			l = mid+1;
		else
			r = mid;
	}
	pl = l;

	l = pos;
	r = len-1;
	while (l < r)
	{
		mid = l+r+1>>1;
		cp = lcp(pos,mid);
		if (cp < comlen)
			r = mid-1;
		else
			l = mid;
	}
	pr = l;
}
		\end{lstlisting}