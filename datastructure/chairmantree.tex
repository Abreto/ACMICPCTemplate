\subsection{可持久化线段树}
	区间第$k$小数,内存压缩版,POJ2014。
	\begin{lstlisting}[language=c++]
#include <cstdio>
#include <algorithm>
using namespace std;

const int MAXN=100000,MAXM=100000;

struct node
{
	node *l,*r;
	int sum;
}tree[MAXN*4+MAXM*20];

int N;
node *newnode()
{
	tree[N].l=tree[N].r=NULL;
	tree[N].sum=0;
	return &tree[N++];
}
node *newnode(node *x)
{
	tree[N].l=x->l;
	tree[N].r=x->r;
	tree[N].sum=x->sum;
	return &tree[N++];
}
node *build(int l,int r)
{
	node *x=newnode();
	if (l<r)
	{
		int mid=l+r>>1;
		x->l=build(l,mid);
		x->r=build(mid+1,r);
		x->sum=x->l->sum+x->r->sum;
	}
	else
		x->sum=0;
	return x;
}
node *update(node *x,int l,int r,int p,int v)
{
	if (l<r)
	{
		int mid=l+r>>1;
		node *nx=newnode(x);
		if (p<=mid)
		{
			node *ret=update(x->l,l,mid,p,v);
			nx->l=ret;
		}
		else
		{
			node *ret=update(x->r,mid+1,r,p,v);
			nx->r=ret;
		}
		nx->sum=nx->l->sum+nx->r->sum;
		return nx;
	}
	else
	{
		node *nx=newnode(x);
		nx->sum+=v;
		return nx;
	}
}
int query(node *x1,node *x2,int l,int r,int k)
{
	if (l<r)
	{
		int mid=l+r>>1;
		int lsum=x2->l->sum-x1->l->sum;
		if (lsum>=k)
			return query(x1->l,x2->l,l,mid,k);
		else
			return query(x1->r,x2->r,mid+1,r,k-lsum);
	}
	else
		return l;
}
char s[10];
node *root[MAXM+1];
int a[MAXN],b[MAXN];
int init(int n)
{
	for (int i=0;i<n;i++)
		b[i]=a[i];
	sort(b,b+n);
	int tn=unique(b,b+n)-b;
	for (int i=0;i<n;i++)
	{
		int l=0,r=tn-1;
		while (l<r)
		{
			int mid=l+r>>1;
			if (b[mid]>=a[i])
				r=mid;
			else
				l=mid+1;
		}
		a[i]=l;
	}
	return tn;
}
int main()
{
	int cas=1,n;
	while (scanf("%d",&n)!=EOF)
	{
		printf("Case %d:\n",cas++);
		for (int i=0;i<n;i++)
			scanf("%d",&a[i]);
		int tn=init(n);
		N=0;
		root[0]=build(0,tn-1);
		for (int i=1;i<=n;i++)
			root[i]=update(root[i-1],0,tn-1,a[i-1],1);
		int m;
		scanf("%d",&m);
		for (int i=0;i<m;i++)
		{
			int s,t;
			scanf("%d%d",&s,&t);
			printf("%d\n",b[query(root[s-1],root[t],0,tn-1,t-s+2>>1)]);
		}
	}
	return 0;
}
	\end{lstlisting}