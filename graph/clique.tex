\subsection{最大团以及相关知识}
	\begin{description}
	\item[独立集:] 独立集是指图的顶点集的一个子集,该子集的导出子图不含边.如果一个独立集不是任何一个独立集的子集, 那么称这个独立集是一个极大独立集.一个图中包含顶点数目最多的独立集称为最大独立集。最大独立集 一定是极大独立集,但是极大独立集不一定是最大的独立集。
	\item[支配集:] 与独立集相对应的就是支配集,支配集也是图顶点集的一个子集,设$S$是图$G$的一个支配集,则对于图中的任意一个顶点$u$,要么属于集合$s$, 要么与$s$中的顶点相邻。在$s$中除去任何元素后$s$不再是支配集,则支配集$s$是极小支配集。称$G$的所有支配集中顶点个数最 少的支配集为最小支配集,最小支配集中的顶点个数成为支配数。
	\item[最小点的覆盖:] 最小点的覆盖也是图的顶点集的一个子集,如果我们选中一个点,则称这个点将以他为端点的所有边都覆盖了。将图中所有的边都覆盖所用顶点数最少,这个集合就是最小的点的覆盖。
	\item[最大团:] 图$G$的顶点的子集,设$D$是最大团,则$D$中任意两点相邻。若$u$,$v$是最大团,则$u$,$v$有边相连,其补图$u$,$v$没有边相连,所以图$G$的最大团$=$其补图的最大独立集。给定无向图$G=(V,E)$,如果$U$属于$V$,并且对于任意$u$,$v$包含于$U$ 有$<u,v>$包含于$E$,则称$U$是$G$的完全子图,$G$的完全子图$U$是$G$的团,当且仅当$U$不包含在$G$的更大的完全子图中,$G$的最大团是指$G$中所含顶点数目最多的团。如果$U$属于$V$,并且对于任意$u,v$包含于$U$有$<u,v>$不包含于$E$,则称$U$是$G$的空子图,$G$的空子图U是G的独立集,当且仅当$U$不包含在$G$的更大的独立集,$G$的最大团是指$G$中所含顶点数目最多的独立集。
	\item[一些性质:] 最大独立集$+$最小覆盖集$=V$,最大团$=$补图的最大独立集,最小覆盖集$=$最大匹配
	\end{description}
	
	\begin{lstlisting}[language=c++]
#include <cstdio>
bool am[100][100];
int ans;
int c[100];
int U[100][100];
int n;
bool dfs(int rest,int num)
{
	if (!rest)
	{
		if (num>=ans)
			return 1;
		else
			return 0;
	}
	int pre=-1;
	for (int i=0;i<rest && rest-i+num>=ans;i++)
	{
		int idx=U[num][i];
		if (num+c[idx]<ans)
			return 0;
		int nrest=0;
		for (int j=i+1; j<rest; j++)
			if (am[idx][U[num][j]])
				U[num+1][nrest++]=U[num][j];
		if (dfs(nrest,num+1))
			return 1;
	}
	return 0;
}
int main()
{
	while (scanf("%d",&n),n)
	{
		for (int i=0;i<n;i++)
			for (int j=0;j<n;j++)
				scanf("%d",&am[i][j]);
		ans=0;
		for (int i=n-1; i>=0; i--)
		{
			int rest=0;
			for (int j=i+1; j<n; j++)
				if (am[i][j])
					U[0][rest++]=j;
			ans+=dfs(rest,0);
			c[i]=ans;
		}
		printf("%d\n",ans);
	}
	return 0;
}
	\end{lstlisting}
