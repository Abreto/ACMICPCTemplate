\subsection{baby step giant step}
	\subsubsection{BSGS}
	\begin{lstlisting}[language=c++]
#define MOD 76543
int hs[MOD], head[MOD], next[MOD], id[MOD], top;
void insert(int x, int y)
{
	int k = x%MOD;
	hs[top] = x, id[top] = y, next[top] = head[k], head[k] = top++;
}
int find(int x)
{
	int k = x%MOD;
	for (int i = head[k]; i; i = next[i]) if (hs[i] == x) 
		return id[i];
	return -1;
}
int BSGS(int a, int b, int n)
{
	memset(head, 0, sizeof(head));
	top = 1;
	if (b==1) return 0;
	int m = sqrt(n+.0), j;
	long long x = 1, p = 1;
	for (int i = 0; i < m; ++i, p = p*a%n) insert(p*b%n, i);
	for (long long i = m; ; i += m)
	{
		if ((j = find(x=x*p%n)) != -1) return i-j;
		if (i > n) break;
	}
	return -1;
}
	\end{lstlisting}

	\subsubsection{何老师的版}
	\begin{lstlisting}[language=c++]
//离散对数
#include <cstdio>
#include <cstring>
#include <cmath>
#include <algorithm>
using namespace std;
typedef long long LL;
struct Hash
{
    static const int MOD = 100007;
    static const int MaxN = 100005;
    struct Node
    {
        LL k, v;//A^k = v
        Node *nxt;
    } buf[MaxN], *g[MaxN], *pt;
    void init()
    {
        memset(g,0,sizeof(g)) ;
        pt = buf;
    }
    LL find(LL v)
    {
        for (Node *now = g[v%MOD]; now; now = now->nxt)
            if (now->v == v)
                return now->k;
        return -1;
    }
    void Ins(LL k, LL v)
    {
        if ( find (v) != -1)return;
        pt->k = k;
        pt->v = v;
        pt->nxt = g[v % MOD];
        g[v % MOD] = pt++;
    }
}hash;
LL gcd(LL x, LL y)
{
    return y==0?x:gcd(y,x%y);
}
LL e_gcd(LL a, LL b, LL &x, LL &y)
{
    if (b==0)
    {
        x = 1;
        y = 0;
        return a;
    }
    LL ret = e_gcd(b, a%b, y, x) ;
    y = y - a/b*x;
    return ret ;
}
LL Baby(LL A, LL B, LL C)//A^x = B (mod C)
{
    B %= C;
    A %= C;
    LL x = 1%C, y;
    for (int i = 0; i <= 64; i++)
    {
        if (x==B)return i;
        x = x*A % C;
    }

    LL D = 1%C, g;
    int cnt = 0;
    while((g = gcd(A,C)) != 1)
    {
        if (B%g) return -1;
        cnt++;
        C /= g;
        B /= g;
        D = A/g * D % C;
    }
    hash. init () ;
    int m = (int)sqrt(C);
    LL Am = 1%C;
    hash.Ins(0,Am);
    for (int i = 1; i <= m; i++)
    {
        Am = Am*A % C;
        hash. Ins ( i ,Am);
    }
    for (int i = 0; i <= m; i++)
    {
		//D*x = B (mod C), D*x + C*y = B
        g = e_gcd(D,C,x,y);
        x = (x*B/g%C+C)%C;
        LL k = hash.find(x) ;
        if (k != -1) return i*m+k+cnt;
        D = D*Am % C;
    }
    return -1;
}
int main()
{
    int A,B,C;
    while(scanf("%d%d%d",&A,&C,&B) == 3 && (A+B+C))
    {
        if (B>=C)
        {
            puts("Orz,I can’t find D!");
            continue;
        }
        LL ret = Baby(A,B,C);
        if ( ret == -1)puts("Orz,I can’t find D!");
        else printf ("%I64d\n",ret);
    }
    return 0;
}
	\end{lstlisting}