\subsection{稳定婚姻}
	假定有$n$个男生和$m$个女生,理想的拍拖状态就是对于每对情侣$(a,b)$,找不到另一对情侣$(c,d)$使得$c$更喜欢$b$,$b$也更喜欢$c$,同理,对$a$来说也没有$(e,f)$使得$a$更喜欢$e$而$e$更喜欢$a$,当然最后会有一些人落单。这样子一个状态可以称为理想拍拖状态,它也有一个专业的名词叫稳定婚姻。\\
	求解这个问题可以用一个专有的算法,延迟认可算法,其核心就是让每个男生按自己喜欢的顺序逐个向女生表白,例如leokan向一个女生求爱,这个过程中,若这个女生没有男朋友,那么这个女生就暂时成为leokan的女朋友,或这个女生喜欢她现有男朋友的程度没有喜欢leokan高,这个女生也暂时成为leokan的女朋友,而她原有的男朋友则再将就找下一个次喜欢的女生来当女朋友。
	\begin{lstlisting}[language=c++]
#include<string.h>
#include<stdio.h>
#define N 1050
int boy[N][N];
int girl[N][N];
int ans[N];
int cur[N];
int n;
void getMarry(int g)
{
	for (int i=ans[g]+1;i<n;i++)
	{
		int b=girl[g][i]-1;
		if (cur[b]<0)
		{
			ans[g]=i;
			cur[b]=g;
			return;
		}
		int og=cur[b];
		if (boy[b][og] > boy[b][g])
		{
			cur[b]=g;
			ans[g]=i;
			getMarry(og);
			return;
		}
	}
};
int main()
{
	int t,a;
	scanf("%d",&t);
	while(t--)
	{
		memset(girl,0,sizeof(girl));
		memset(boy,0,sizeof(boy));
		scanf("%d",&n);
		for (int i=0;i<n;i++)
			for (int j=0;j<n;j++)
				scanf("%d",&girl[i][j]);
		for (int i=0;i<n;i++)
			for (int j=0;j<n;j++)
			{
				scanf("%d",&a);
				boy[i][a-1]=j;
			}
		memset(cur,0xff,sizeof(cur));
		memset(ans,0xff,sizeof(ans));
		for (int i=0;i<n;i++)
			getMarry(i);
		for (int i=0;i<n;i++)
			printf("%d\n",girl[i][ans[i]]);
	}
	return 0;
}
	\end{lstlisting}