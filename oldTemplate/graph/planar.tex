\subsection{*二维平面图的最大流}
	待整理\\
	\begin{lstlisting}[language=c++]
#include <iostream>
#include <algorithm>
#include <cstdio>
#include <cstring>
#include <vector>
#include <cmath>
#include <map>
#include <queue>
using namespace std;

const int maxn = 100100;
const int inf = 0x3f3f3f3f;
struct Point
{
	int x,y,id;
	double theta;
	Point() {}
	Point(int _x,int _y)
	{
		x = _x;
		y = _y;
	}
	Point(Point _s,Point _e,int _id)
	{
		id = _id;
		x = _s.x-_e.x;
		y = _s.y-_e.y;
		theta = atan2(y,x);
	}
	bool operator < (const Point &b)const
	{
		return theta < b.theta;
	}
};

map<pair<int,int>,int > idmap;
struct Edge
{
	int from,to,next,cap,near,mark;
};
Edge edge[maxn*2];
int head[maxn],L;
int cntd[maxn];
void addedge(int u,int v,int cap)
{
	cntd[u]++;
	cntd[v]++;
	idmap[make_pair(u,v)] = L;
	edge[L].from = u;
	edge[L].to = v;
	edge[L].cap = cap;
	edge[L].next = head[u];
	edge[L].mark = -1;
	head[u] = L++;
}

int rtp[maxn];
Point p[maxn],tp[maxn];
int n,m,S,T;
int vid;

struct Edge2
{
	int to,next,dis;
} edge2[maxn*2];
int head2[maxn],L2;

void addedge2(int u,int v,int dis)
{
	edge2[L2].to = v;
	edge2[L2].dis = dis;
	edge2[L2].next = head2[u];
	head2[u] = L2++;
}

int dist[maxn];
bool inq[maxn];
int SPFA(int s,int t)
{
	queue<int> Q;
	memset(inq,false,sizeof(inq));
	memset(dist,63,sizeof(dist));
	Q.push(s);
	dist[s] = 0;
	while (!Q.empty())
	{
		int now = Q.front();
		Q.pop();
		for (int i = head2[now]; i != -1; i = edge2[i].next)
			if (dist[edge2[i].to] > dist[now]+edge2[i].dis)
			{
				dist[edge2[i].to] = dist[now]+edge2[i].dis;
				if (inq[edge2[i].to] == false)
				{
					inq[edge2[i].to] = true;
					Q.push(edge2[i].to);
				}
			}
		inq[now] = false;
	}
	return dist[t];
}

int main()
{
	int totcas;
	scanf("%d",&totcas);
	for (int cas = 1; cas <= totcas; cas++)
	{
		idmap.clear();
		L = 0;
		scanf("%d%d",&n,&m);
		S = T = 0;
		for (int i = 0; i < n; i++)
		{
			head[i] = -1;
			scanf("%d%d",&p[i].x,&p[i].y);
			if (p[S].x > p[i].x)
				S = i;
			if (p[T].x < p[i].x)
				T = i;
			cntd[i] = 0;
		}
		//源汇中间加入一个特殊节点
		head[n] = -1;
		n ++;
		addedge(S,n-1,inf);
		addedge(n-1,S,inf);
		addedge(T,n-1,inf);
		addedge(n-1,T,inf);

		for (int i = 0; i < m; i++)
		{
			int u,v,cap;
			scanf("%d%d%d",&u,&v,&cap);
			u--;
			v--;
			addedge(u,v,cap);
			addedge(v,u,cap);
		}

		for (int i = 0; i < n; i++)
		{
			int tot = 0;
			//源点汇点连到特殊点的方向需要特别考虑一下
			if (i == S)
				tp[tot++] = Point(Point(0,0),Point(-1,0),n-1);
			else if (i == T)
				tp[tot++] = Point(Point(0,0),Point(1,0),n-1);
			else if (i == n-1)
			{
				tp[tot++] = Point(Point(0,0),Point(1,0),S);
				tp[tot++] = Point(Point(0,0),Point(-1,0),T);
			}
			if (i < n-1)
			{
				for (int j = head[i]; j != -1; j = edge[j].next)
				{
					if (i == S && edge[j].to == n-1)  continue;
					if (i == T && edge[j].to == n-1)  continue;
					tp[tot++] = Point(p[i],p[edge[j].to],edge[j].to);
				}
			}
			sort(tp,tp+tot);
			for (int j = 0; j < tot; j++)
				rtp[tp[j].id] = j;
			for (int j = head[i]; j != -1; j = edge[j].next)
				edge[j].near = tp[(rtp[edge[j].to]+1)%tot].id;
		}

		vid = 0;
		for (int i = 0;i < L;i++)
			if (edge[i].mark == -1)
			{
				int now = edge[i].from;
				int eid = i;
				int to  = edge[i].to;
				while (true)
				{
					edge[eid].mark = vid;
					eid ^= 1;
					now = to;
					to = edge[eid].near;
					eid = idmap[make_pair(now,to)];

					if (now == edge[i].from)	break;
				}
				vid++;
			}

		L2 = 0;
		for (int i = 0; i < vid; i++)
			head2[i] = -1;
		for (int i = 0; i < L; i++)
			addedge2(edge[i].mark,edge[i^1].mark,edge[i].cap);
		printf("%d\n",SPFA(edge[0].mark,edge[1].mark));
	}
	return 0;
}
	\end{lstlisting}