\subsection{圆}	
	\subsubsection{面积:两圆相交}
	圆不可包含
	\begin{lstlisting}[language=c++]
double dis(int x,int y)
{
	return sqrt((double)(x*x+y*y));
}
double area(int x1,int y1,int x2,int y2,double r1,double r2)
{
	double s=dis(x2-x1,y2-y1);
	if(r1+r2<s) return 0;
	else if(r2-r1>s) return PI*r1*r1;
	else if(r1-r2>s) return PI*r2*r2;
	double q1=acos((r1*r1+s*s-r2*r2)/(2*r1*s));
	double q2=acos((r2*r2+s*s-r1*r1)/(2*r2*s));
	return (r1*r1*q1+r2*r2*q2-r1*s*sin(q1));
}
	\end{lstlisting}
	
	\subsubsection{三角形外接圆}
	\begin{lstlisting}[language=c++]
void CircumscribedCircle()
{
	for (int i = 0; i < 3; i++)
		scanf("%lf%lf",&p[i].x,&p[i].y);
	tp = Point((p[0].x+p[1].x)/2,(p[0].y+p[1].y)/2);
	l[0] = Line(tp,Point(tp.x-(p[1].y-p[0].y),tp.y+(p[1].x-p[0].x)));
	tp = Point((p[0].x+p[2].x)/2,(p[0].y+p[2].y)/2);
	l[1] = Line(tp,Point(tp.x-(p[2].y-p[0].y),tp.y+(p[2].x-p[0].x)));
	tp = LineToLine(l[0],l[1]);
	r = Point(tp,p[0]).Length();
	printf("(%.6f,%.6f,%.6f)\n",tp.x,tp.y,r);
}
	\end{lstlisting}
	
	\subsubsection{三角形内切圆}
	\begin{lstlisting}[language=c++]
void InscribedCircle()
{
	for (int i = 0; i < 3; i++)
		scanf("%lf%lf",&p[i].x,&p[i].y);
	if (xmult(Point(p[0],p[1]),Point(p[0],p[2])) < 0)
		swap(p[1],p[2]);
	for (int i = 0; i < 3; i++)
		len[i] = Point(p[i],p[(i+1)%3]).Length();
	tr = (len[0]+len[1]+len[2])/2;
	r = sqrt((tr-len[0])*(tr-len[1])*(tr-len[2])/tr);
	for (int i = 0; i < 2; i++)
	{
		v = Point(p[i],p[i+1]);
		tv = Point(-v.y,v.x);
		tr = tv.Length();
		tv = Point(tv.x*r/tr,tv.y*r/tr);
		tp = Point(p[i].x+tv.x,p[i].y+tv.y);
		l[i].s = tp;
		tp = Point(p[i+1].x+tv.x,p[i+1].y+tv.y);
		l[i].e = tp;
	}
	tp = LineToLine(l[0],l[1]);
	printf("(%.6f,%.6f,%.6f)\n",tp.x,tp.y,r);
}
	\end{lstlisting}
	
	\subsubsection{点对圆的两个切点}
	\begin{lstlisting}[language=c++]
void calc_qie(Point poi,Point o,double r,Point &result1,Point &result2)
{
	double line = sqrt((poi.x-o.x)*(poi.x-o.x)+(poi.y-o.y)*(poi.y-o.y));
	double angle = acos(r/line);
	Point unitvector,lin;
	lin.x = poi.x-o.x;
	lin.y = poi.y-o.y;
	unitvector.x = lin.x/sqrt(lin.x*lin.x+lin.y*lin.y)*r;
	unitvector.y = lin.y/sqrt(lin.x*lin.x+lin.y*lin.y)*r;
	result1 = unitvector.Rotate(-angle);
	result2 = unitvector.Rotate(angle);
	result1.x += o.x;
	result1.y += o.y;
	result2.x += o.x;
	result2.y += o.y;
}
	\end{lstlisting}
	
	\subsubsection{两圆公切点}
	\begin{lstlisting}[language=c++]
void Gao()
{
	tn = 0;
	Point a,b,vab;
	double tab,tt,dis,theta;
	for (int i = 0; i < tc; i++)
		for (int j = 0; j < tc; j++)
			if (i != j)
			{
				a = c[i];
				b = c[j];
				vab = Point(a,b);
				tab = atan2(vab.y,vab.x);
				dis = sqrt(vab.x*vab.x+vab.y*vab.y);
				if (b.r > a.r)
					tt = asin((b.r-a.r)/dis);
				else
					tt = -asin((a.r-b.r)/dis);
				theta = tab+pi/2+tt;
				tp[tn++] = Point(a.x+a.r*cos(theta),a.y+a.r*sin(theta));
				tp[tn++] = Point(b.x+b.r*cos(theta),b.y+b.r*sin(theta));
			}
}
	\end{lstlisting}
	
	\subsubsection{两圆交点}
	\begin{lstlisting}[language=c++]
lab = Point(p[j].x-p[i].x,p[j].y-p[i].y);
AB = lab.Length();
AC = cr[i];
BC = cr[j];

if (cmp(AB+AC,BC) <= 0)	continue;//包含
if (cmp(AB+BC,AC) <= 0)	continue;
if (cmp(AB,AC+BC) > 0)	continue;//相离

theta = atan2(lab.y,lab.x);
fai = acos((AC*AC+AB*AB-BC*BC)/(2.0*AC*AB));
a0 = theta-fai;
if (cmp(a0,-pi) < 0)	a0 += 2*pi;
a1 = theta+fai;
if (cmp(a1,pi) > 0)	a1 -= 2*pi;
//答案
xp[totp++] = Point(p[i].x+cr[i]*cos(a0),p[i].y+cr[i]*sin(a0));
xp[totp++] = Point(p[i].x+cr[i]*cos(a1),p[i].y+cr[i]*sin(a1));
	\end{lstlisting}